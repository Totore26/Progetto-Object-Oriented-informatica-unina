\section{Procedure}
\subsection{avviso\_su\_impiegati\_licenziati}
\begin{itemize}
\normalsize
\item La procedura \textbf{avviso\_su\_impiegati\_licenziati} ha lo scopo di verificare quali impiegati sono arrivati al termine del contratto (sono giunti alla data\_licenziamento) e , nel caso in cui vi sono degli impiegati con ruoli speciali, avvisa l'utente di aggiornare tali ruoli.

\scriptsize
\begin{lstlisting}
CREATE OR REPLACE PROCEDURE avviso_su_impiegati_licenziati() AS
$$ DECLARE
cursore_rscientifico CURSOR IS (select i.matricola, l.id_lab from impiegato AS i JOIN laboratorio AS l ON i.matricola = l.r_scientifico where i.matricola NOT IN (select matricola from Impiegati_attuali));
cursore_referenti CURSOR IS (select i.matricola, pa.cup from impiegato AS i JOIN PROGETTI_ATTUALI AS pa ON (i.matricola = pa.referente) where i.matricola NOT IN (select matricola from Impiegati_attuali));
cursore_responsabili CURSOR IS (select i.matricola, pa.cup from impiegato AS i JOIN PROGETTI_ATTUALI AS pa ON (i.matricola = pa.responsabile) where i.matricola NOT IN (select matricola from Impiegati_attuali));
rigacorrente RECORD;
BEGIN
	OPEN cursore_rscientifico;
	LOOP
		FETCH cursore_rscientifico INTO rigacorrente;
		IF NOT FOUND THEN
			EXIT;
		END IF;
		RAISE WARNING 'L IMPIEGATO CON MATRICOLA % E STATO LICENZIATO, AGGIORNA IL REFERENTE SCIENTIFICO NEL LABORATORIO %',
						rigacorrente.matricola, rigacorrente.id_lab;
	END LOOP;
	CLOSE cursore_rscientifico;

	OPEN cursore_referenti;
	LOOP
		FETCH cursore_referenti INTO rigacorrente;
		IF NOT FOUND THEN
			EXIT;
		END IF;
		RAISE WARNING 'L IMPIEGATO CON MATRICOLA % E STATO LICENZIATO, AGGIORNA IL REFERENTE NEL PROGETTO %',
						rigacorrente.matricola, rigacorrente.cup;
	END LOOP;
	CLOSE cursore_referenti;

	OPEN cursore_responsabili;
	LOOP
		FETCH cursore_responsabili INTO rigacorrente;
		IF NOT FOUND THEN
			EXIT;
		END IF;
		RAISE WARNING 'L IMPIEGATO CON MATRICOLA % E STATO LICENZIATO, AGGIORNA IL RESPONSABILE NEL PROGETTO %',
						rigacorrente.matricola, rigacorrente.cup;
	END LOOP;
	CLOSE cursore_responsabili;
END; $$ LANGUAGE plpgsql;
\end{lstlisting}
%MANCA LA PROCEDURA AGGIORNA_DATABASE E POI FINITO COME SBORRO
\newpage
\subsection{update\_database}
\normalsize
\item La procedura \textbf{update\_database} è la routine di base per aggiornare automaticamente gli scatti di carriera di tutti gli impiegati, aggiornando anche l'attributo tipo\_impiegato se necessario.
\scriptsize
\begin{lstlisting}
CREATE OR REPLACE PROCEDURE update_database() AS
$$ DECLARE
	cursore_impiegati cursor is (select* from Impiegati_attuali as i natural join storico as s where s.nuovo_ruolo = i.tipo_impiegato);
	imp_corrente record;
BEGIN
	open cursore_impiegati;
	LOOP
	FETCH cursore_impiegati INTO imp_corrente;
	IF NOT FOUND THEN
		EXIT;
	end if;
		IF(imp_corrente.tipo_impiegato = 'junior') THEN
			IF(imp_corrente.data_assunzione + INTERVAL '3 years' <= CURRENT_DATE) THEN
				INSERT INTO STORICO VALUES('junior','middle',imp_corrente.data_assunzione + INTERVAL '3 YEARS', imp_corrente.matricola);
				IF(imp_corrente.data_assunzione + INTERVAL '7 years' >= CURRENT_DATE) THEN
					UPDATE IMPIEGATO
					SET tipo_impiegato = 'middle'
					WHERE impiegato.matricola = imp_corrente.matricola;
				ELSE
					INSERT INTO STORICO VALUES('middle','senior',imp_corrente.data_assunzione + INTERVAL '7 YEARS', imp_corrente.matricola);
					UPDATE IMPIEGATO
					SET tipo_impiegato = 'senior'
					WHERE impiegato.matricola = imp_corrente.matricola;
				END IF;
			END IF;
		END IF;
		IF(imp_corrente.tipo_impiegato = 'middle') THEN
			IF(imp_corrente.data_assunzione + INTERVAL '7 years' <= CURRENT_DATE) THEN
				INSERT INTO STORICO VALUES('middle','senior',imp_corrente.data_assunzione + INTERVAL '7 YEARS', imp_corrente.matricola);
				UPDATE IMPIEGATO
				SET tipo_impiegato = 'senior'
				WHERE impiegato.matricola = imp_corrente.matricola;
			END IF;
		END IF;
	END LOOP;
	close cursore_impiegati;
END; $$ LANGUAGE plpgsql;
\end{lstlisting}

\end{itemize}